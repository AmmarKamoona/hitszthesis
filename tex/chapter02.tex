% !TEX root = ../main.tex

% 第2章的标题
\chapter{文献综述}

% 第2章第1节的标题
\section{实值神经网络发展历程}

% <\upcite{}>为在右上角引用参考文献命令
% <\cite{}>为在正文中引用参考文献命令
% 目前不支持[1-6]类型的引用,需手动填写<$^{[1\text{-}6]}$>

人类利用人工神经网络对生物神经网络的结构、机理和功能模拟的研究,经历了近半
个多世纪漫长而又曲折的过程。具有里程碑意义的是如下三个模型\upcite{bibc1,bibc2}。

1943年神经生理学家McCulloch和数学家Pitts提出的第一个人工神经网络模型:
McCulloch-Pitts模型\upcite{bibc3},成为神经网络研究的开端。1949年心理学家D.O.Hebb在文献\upcite{bibc4}中提出了神经元之间突触联系强度可以改变的假设,并据此提出了神经元学习的准则,为神经网络的学习算法奠定了基础。这些重大进展促进了神经网络早期的研究。

1958年Rosenblat将McCulloch和Pitts的工作与Hebb的工作结合在一起,设计出的
一种称为感知机的神经网络模型\upcite{bibc5},首次把人工神经网络的研究从理论付诸工程实践,引起了许多科学家的兴趣。不过在1969年,人工智能创始人之一Minsky和Papert出版了以《感知机》为名的书\upcite{bibc6},书中从数学上深入分析了感知机的原理,指出了其局限性并从理论上对感知机的原理给予了沉重的打击,使得神经网络早期的研究热潮迅速降温,并开始陷入低潮和休眠状态。

% <\/>命令解决fi、ff等上边连接在一起的问题
1982年和1984年美国加州理工大学的生物物理学家J.J.Hopf\/ield提出了Hopf\/ield神经网络\upcite{bibc7,bibc8}。Hopf\/ield神经网络模型的提出具有划时代的伟大意义,使得神经网络的研究从体眠的状态开始复苏。Hopf\/ield在文献\cite{bibc7,bibc8}中设计出用电子电路实现这一网络的方案,从而为神经计算机的研究奠定了基础。更为重要的是他把设计出的神经网络应用于联想记忆以及解决优化领域一直困扰的TSP问题,并取得了巨大的成功,这些重大进展大大促进了神经网络的研究。此外,他还通过引入类似于Lapunov函的“计算能量函数”的概念对网络的动力学特性进行了分析并给出了网络稳定性的条件,从而开启了神经网络稳定性研究。

% 节标题
\section{复值神经网络的兴起}

复值神经网络(Complex-valued Neural Network)是实值神经网络在复数域的推广。相比实值神经网络,复值神经网络的起步稍晚。其历史可追溯到1975年Widrow提出的复值最小均方(LMS)算法,该算法后来被广泛运用于信号处理领域\upcite{bibc9}。但正如复数的出现和被人认可经历了一个漫长的过程一样,复值神经网络在出现后的相当长时间,并未引起人们的重视。直至近几年来,随着神经网络应用领域的不断扩展,许多需要处理复数信号的领域也提出了用神经网络解决其相关问题的要求,例如量子力学、电磁学、光电子、图像、遥感、时间序列分析等。在这些领域中,基于复数信号的表示、分析和处理具有极大的便利,而复值神经网络则能直接处理复数信号,因而很自然地使二者联系在一起。事实上,在需要幅值和相位(或理解为大小和方向)两方面信息来表示同一信号的场合,运用复值神经网络较之实值神经网络,要更为直接。尽管可以用2个实值神经网络分别表达此类信号的大小和方向(或实部和虚部),但正如复数虽然只是两个实数按某种规则而连接,但依然有其独立于实数而存在的必要性一样,复值神经网络由于复数有实部和虚部,且复数运算规则的特殊性等原因,在其实现和运用过程中表现出一些实值神经网络所不具有特质,因此复值神经网络也不能简单地理解为可用2个实值神经网络所取代。而且在将神经网络从实数域到复数域的推广中,并不是像表面上理解的那样:仅是简单地将一个变量从实数变成复数。如果这样做的话,许多诸如奇异等意想不到的问题就有可能发生\upcite{bibc10}。

作为复值神经网络研究领域的先导者,日本的Akira Hirose教授总结了近年来复值神经网络的研究成果,并出版一本名为《复值神经网络的理论及应用》专著\upcite{bibc10}。书中总结了近年来复值神经网络理论的发展和应用情况。

% 节标题
\section{抓取物体图像识别}

近年来,基于深度学习的方法已经在包括视觉识别\upcite{bib1,bib2},语音识别和自然语言处理等任务上取得了重大成就。美国康奈尔大学 Lenz 等\upcite{bib3}借鉴深度学习在图像检测及图像识别等任务中的成功经验,提出了基于深度学习的机器人抓取检测的方法\upcite{bib3,bib4}。与传统的依靠人工经验抽取样本点特征的方式相比,基于深度学习的机器人抓取检测的方法可以自动学习如何识别和提取待抓取位置的特定特征。越来越多的机器人学者研究如何将深度学习的方法应用于机器人抓取检测上从而使机器人具备更强大的智能。大部分研究学者都是将深度学习的方法用来学习不同形状和位姿的物体的末端执行器的最佳配置,基于深度学习的深层表达能力学习的参数为每个图像预测多个抓取位置进行排序来找到最佳抓取位置。

基于深度学习的方法,Lenz 等\upcite{bib3}提出了基于稀疏自编码器的两步级联抓取检测系统,构建两个大小网络用于提取 RGB-D 输入数据的抓取特征,采用滑动窗的方法搜索抓取框,最后在网络顶层添加支持向量机(Support Vector Machine,SVM)作为分类器的网络结构。在标准康奈尔抓取数据集\upcite{bib6}上达到 73.9\%的检测准确率,耗时13.5s,由于采用类似于穷举法的搜索机制,需要在不同大小的图像块上使用分类器进行重复计算,计算量非常大,且十分耗时。

Redmon 等\upcite{bib7}认为采用滑动窗口的方法来预测抓取位置是一种非常耗时的方法,而且使用单阶段的网络性能优于 Lenz 等的级联系统。为了避免在不同大小的图像块上重复计算,他们利用卷积神经网络强大的特征提取能力将整个图像输入网络中,在整个图像上直接进行全局的抓取预测,目前大部分学者采用这种方案\upcite{bib5,bib8}。使用类似于 AlexNet\upcite{bib9}的卷积神经网络模型来实现单阶段的检测方法,以更快的速度达到了更高的检测精度,但是这种方法由于卷积神经网络结构的复杂性仍然存在模型较大的缺陷。 

Kumra 等\upcite{bib4}也采用将整个图像输入卷积神经网络中进行全局的抓取预测,网络结构上,他们采用网络结构更复杂特征提取能力更强的ResNet50提取抓取特征\upcite{bib10},用 SVM 预测抓取配置的参数。精度上可以达到比较好的检测精度,但是由于模型采用层数较深的残差网络,导致网络模型和计算量都比较大。

Chu 等\upcite{bib5}提出了一种适合于多物体场景抓取模型,首先使用 ResNet50 对输入图像提取抓取特征,然后使用类似于 RPN  网络的模型进行抓取框的推荐,最后经 ROI  Pooling 进行角度参数的分类和抓取框的回归。这种模型适用于多物体的抓取场景, 并且达到了较高的抓取检测准确率,由于模型较深且类似于级联系统导致模型较大。

% 节标题
\section{机器人抓取策略}

在机器人分拣、 搬运等抓取作业任务中,包括顶抓(top-grasp)和侧抓(side-grasp) 2 种方式的平面抓取(planar grasp)是最为常用的抓取策略。对于任意姿态的未知不规则物体,在光照不均、 背景复杂的场景下,如何利用低成本的单目相机实现快速可靠的机器人自主抓取姿态检测具有很大的挑战。

机器人自主抓取姿态规划方法根据感知信息的不同可分为 2 类:一类是基于物体模型的抓取姿态估计\upcite{bibb1,bibb2,bibb3},一类是不依赖物体模型的抓取姿态检测。基于模型的方法需要给定精确、 完整的物体 3 维模型,然而低成本相机的成像噪声大,很难扫描建立精确模型。另外,基于 3 维模型的方法计算复杂,难以适应机器人实时抓取判断的需求。

不依赖物体模型的方法借助于机器学习技术,其实质是将抓取位姿检测问题转化成目标识别问题。例如,文\cite{bibb4} 提出了一种用 2 维矢量矩形表示图像上物体抓取位姿的直观方法。机器学习方法的出现令抓取检测不局限于已知物体。早期的学习方法\upcite{bibb4} 需要人为针对特定物体设定特定的视觉特征,不具备灵活性。近年来,深度学习\upcite{bibb5} 发展迅速,其优越性正在于可自主提取与抓取位姿有关的特征。 