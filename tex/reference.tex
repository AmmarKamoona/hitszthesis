% !TEX root = ../main.tex

% 参考文献
% 文献个数小于99
\begin{thebibliography}{99}
\addcontentsline{toc}{chapter}{参考文献}

% 每条参考文献均需用<\bibitem{}>引出,
% 花括号里的内容为此条参考文献的标签label,
% 可用<\upcite{}>和<\cite{}>引用。

\bibitem{bibc1} 焦李成.神经网络系统理论[M].西安:西安电子科技大学出版社,1991.
\bibitem{bibc2} 何玉彬,李新忠.神经网络控制及其应用[M].北京:科学出版社,2000.
\bibitem{bibc3} McCulloch W S, Pitts W A. A logical calculus of the ideas immanent in nervous activity[J]. Bulletin of Mathematical Biophysics, 1943, 5: 115-133.
\bibitem{bibc4} Hebb D O. The Organization of Behaviour [M]. New York, John Wiley\&Sons Inc., 1949.
\bibitem{bibc5} Rosenblatt. The perception: a probabilistic model for information storage and organization in the brain [J]. Psychology Review, 1958, 65: 386-408.
\bibitem{bibc6} Minsky M, Papert S. Perceptron [M]. Cambridge, MA: MIT Press, 1969.
\bibitem{bibc7} Hopf\/ield  J  J.  Neural  networks  and  physical  systems  with  emergent  collective computational  abilities[C].  Proceeding  of the National Academy  of  Science.  USA (Biophysics), 1982, 79: 2554-2558.
\bibitem{bibc8} Hopf\/ield J J. Neurons with graded response have collective computational properties like those  of  two-state  neurons[C].  Proceedins  of  the  National  Academy  of  Science, USA(Biophysics), 1984, 81:3088-3092.
\bibitem{bibc9} Widrow B, McCool J, Ball M. The complex LMS algorithm [C]. Proc. IEEE, 1975, 63(4):719-720
\bibitem{bibc10} Hirose  A.  Complex-valued  neural  networks:  theories  and  applications  [M].  World Scientif\/ic Series on Innovation Intelligence, vol 5, Singapore: World Scientif\/ic Publishing Co. Pte. Ltd. 2003

\bibitem{bib1} 谢林江, 季桂树, 彭清, 等. 改进的卷积神经网络在行人检测中的应用[J].  计算机科学与探索,  2018,  12(5):708-718.
\bibitem{bib2}王耀玮, 唐伦, 刘云龙, 等. 基于多任务卷积神经网络的车辆多属性识别[J].  计算机工程与应用, 2018, 54(8):21-27.
\bibitem{bib3} Lenz I, Lee H, Saxena A. Deep learning for detecting robotic grasps[J]. The International Journal of Robotics Research, 2015, 34(4-5):705-724.
\bibitem{bib4} Kumra S, Kanan C. Robotic grasp detection using deep convolutional neural networks[C]//2017 IEEE/RSJ International Conference on Intelligent Robots and Systems (IROS). IEEE, 2017: 769-776.
\bibitem{bib5} Chu F J, Xu R, Vela P A. Real-world multiobject, multigrasp detection[J]. IEEE Robotics and Automation Letters, 2018, 3(4): 3355-3362.
\bibitem{bib6}  Robot Learning Lab: Learning to Grasp[EB/OL].(2009) [2019-03-12].
\bibitem{bib7}  Redmon J, Angelova A. Real-time grasp detection using convolutional neural networks[C]//2015 IEEE International Conference on Robotics and Automation (ICRA). IEEE, 2015: 1316-1322.
\bibitem{bib8} Ni P, Zhang W, Bai W, et al. A New Approach Based on Two-stream CNNs for Novel Objects Grasping in Clutter[J]. Journal of Intelligent \& Robotic Systems, 2018(2):1-17.
\bibitem{bib9} Krizhevsky A, Sutskever I, Hinton G E. Imagenet classif\/ication with deep convolutional neural networks[C]// Advances in neural information processing systems. 2012: 1097-1105.
\bibitem{bib10} He K, Zhang X, Ren S, et al. Deep residual learning for image recognition[C]//Proceedings of the IEEE conference on computer vision and pattern recognition. 2016: 770-778.

\bibitem{bibb1} Dogar M, Hsiao K, Ciocarlie M, et al. Physics-based grasp planning through clutter[C]//Robotics: Science and Systems VIII. Cambridge, USA: MIT Press, 2012: 8pp.
\bibitem{bibb2} Goldfeder C, Ciocarlie M, Dang H, et al. The Columbia grasp database[C]//IEEE International Conference on Robotics and Automation. Piscataway, USA: IEEE, 2009: 1710-1716.
\bibitem{bibb3} Weisz J, Allen P K. Pose error robust grasping from contact wrench space metrics[C]//IEEE International Conference on Robotics and Automation. Piscataway, USA: IEEE, 2012: 557- 562.
\bibitem{bibb4}   Jiang Y, Moseson S, Saxena A. Eff\/icient grasping from RGB-D images: Learning using a new rectangle representation[C]// IEEE International Conference on Robotics and Automation. Piscataway, USA: IEEE,2011: 3304-3311.
\bibitem{bibb5}  Hinton G E, Salakhutdinov R R. Reducing the dimensionality of data with neural networks[J]. Science, 2006, 313(5786): 504-507.

\bibitem{bib11} Qiang Z, Li Z, Li J, et al. Vehicle color recognition using
Multiple-Layer Feature Representations of lightweight convolutional neural network[J]. Signal Processing, 2018, 147: 146-153. 

 \bibitem{bib:one} 马倩倩,李晓娟,施智平.轻量级卷积神经网络的机器人抓取检测研究[J/OL].计算机工程与应用:1-11[2019-04-09].
 
 \bibitem{bib12} Iandola F N, Han S, Moskewicz M W, et al. SqueezeNet: AlexNet-level accuracy with 50x fewer parameters and<
0.5 MB model size[J].2016.

 \bibitem{bib13} Huang G, Liu Z, Van Der Maaten L, et al. Densely connected convolutional networks[C]//Proceedings of the IEEE conference on computer vision and pattern recognition, 2017: 4700-4708.

 \bibitem{bib14} Szegedy C, Vanhoucke V, Ioffe S, et al. Rethinking the inception architecture for computer vision[C]// Proceedings of the IEEE conference on computer vision and pat- tern recognition. 2016: 2818-2826.
 \bibitem{bib15} Huang G, Liu Z, Van Der Maaten L, et al. Densely connected convolutional networks[C]//Proceedings of the IEEE conference on computer vision and pattern recognition. 2017: 4700-4708.
 \bibitem{bib16} He K, Sun J. Convolutional neural networks at con-strained time cost[C]//Proceedings of the IEEE conference on computer vision and pattern recognition. 2015: 5353-5360.
 \bibitem{bib17} Srivastava N, Hinton G, Krizhevsky A, et al. Dropout: a simple way to prevent neural networks from overfitting[J]. The Journal of Machine Learning Research, 2014, 15(1): 1929-1958.

\bibitem{bibb6}   仲训杲,徐敏,仲训昱,等.基于多模特征深度学习的机器人抓取判别方法 [J].自动化学报,2016,42(7):1022- 1029.
\bibitem{bibb7} Lenz I, Lee H, Saxena A. Deep learning for detecting robotic grasps[J]. International Journal of Robotics Research, 2015, 34(4/5): 705-724.
\bibitem{bibb8}   杜学丹,蔡莹皓,鲁涛,等.一种基于深度学习的机械臂抓取方法  [J].机器人,2017,39(6):820-828,837.
\bibitem{bibb9} Pinto L, Gupta A. Supersizing self-supervision: Learning to grasp from 50k tries and 700 robot hours[C]//IEEE International Conference on Robotics and Automation. Piscataway, USA: IEEE, 2016: 3406-3413.
\bibitem{bibb10} Guo D, Sun F C, Liu H P, et al. A hybrid deep architecture for robotic grasp detection[C]//IEEE International Conference on Robotics and Automation. Piscataway, USA: IEEE, 2017: 1609-1614.
\bibitem{bibb11} 刘义军.基于 FPGA 的线结构光视觉测量系统研究 [D]. 长 春:吉林大学,2017:23-49.

 \bibitem{bib:three} 夏晶,钱堃,马旭东,刘环.基于级联卷积神经网络的机器人平面抓取位姿快速检测[J].机器人,2018,40(06):794-802.
 
\bibitem{bibb12} 邹媛媛,赵明扬,张雷.基于量块的线结构光视觉传感器直接标定方法 [J]. 中国激光,2014,41(11):189-194.
\bibitem{bibb13} 卢津,孙惠斌,常智勇.新型正交消隐点的摄像机标定方法 [J]. 中国激光,2014,41(2):294-302.

\bibitem{bibb14} Huang J, Rathod V, Sun C, et al. Speed/accuracy trade-offs for modern convolutional object detectors[A/OL]. (2017-04-25) [2017-12-07].    
\bibitem{bibb15} Girshick R, Donahue J, Darrell T, et al. Rich feature hierarchies for accurate object detection and semantic segmentation[C]//IEEE Conference on Computer Vision and Pattern Recognition. Piscataway, USA: IEEE, 2014: 580-587.
\bibitem{bibb16} Girshick R. Fast R-CNN[C]//IEEE International Conference on Computer Vision. Piscataway, USA: IEEE, 2015: 1440-1448.
\bibitem{bibb17} Ren S Q, He K M, Girshick R, et al. Faster R-CNN: Towards real-time object detection with region proposal networks[M]// Advances in Neural Information Processing Systems 28. Cam- bridge, USA: MIT Press, 2015: 91-99.
\bibitem{bibb18} Liu W, Anguelov D, Erhan D, et al. SSD: Single shot multibox detector[C]//European Conference on Computer Vision. Cham, Switzerland: Springer, 2016: 21-37.
\bibitem{bibb19} Redmon J, Divvala S, Girshick R, et al. You only look once: Unified, real-time object detection[C]//IEEE Conference on Computer Vision and Pattern Recognition. Piscataway, USA:IEEE, 2016: 779-788.
\bibitem{bibb20} Szegedy C, Ioffe S, Vanhoucke V, et al. Inception-v4, Inception-ResNet and the impact of residual connections on learning [A/OL]. (2016-08-23) [2017-12-07].

\bibitem{bib:two} 李传浩. 基于卷积神经网络的机器人自动抓取规划研究[D].哈尔滨工业大学,2018.
 
 \bibitem{bib:four} 胡琳,晁飞.基于双神经网络结构的发展型机器人3D抓取[J].电脑知识与技术,2012,8(12):2859-2862.
\bibitem{bib:five} 刘晓玉. 复杂环境下基于神经网络的工件识别与机器人智能抓取[D].武汉科技大学,2009.
\bibitem{bib:six} 游辉胜. 基于模糊神经网络的单目视觉伺服机器人智能抓取[D].武汉科技大学,2008.
\end{thebibliography}