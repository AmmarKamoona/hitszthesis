% !TEX root = ../main.tex

% 开始写摘要
\begin{abstract}
 
近年来,基于神经网络方法的机器人智能抓取物体的研究方兴未艾。本文首先介绍了该研究的背景,采用神经网络的方法进行机器人智能抓取研究是一个可行的方案。

而后,本文总结归纳了神经网络在机器人智能抓取领域的研究历史与研究现状。由实值神经网络的三个里程碑意义的模型到复值神经网络的兴起,神经网络的模型得到了极大地扩展。基于深度学习的方法已经在机器人视觉识别领域取得了重大的成就,识别要抓取的物体图像以后,机器人的抓取策略也很有讲究。

本文还重点介绍了基于三种不同的神经网络方法进行的机器人智能抓取研究。首先是基于复值神经网络的方法,该方法包含离散、连续、多值三种模型,复值神经网络以其自然的复数处理能力,使得图像等常需频域处理的信号有了直接的表达和处理方式。

然后是基于轻量级卷积神经网络的方法,目前基于神经网络的机器人抓取位姿预测方法的研究使用的网络结构通常具有大量的参数,需要大量的计算和存储资源。基于 SqueezeNet 的轻量级卷积神经网络抓取预测模型在不降低准确率的情况下,网络模型更小,需要的存储资源更少,速度更快。

最后是基于级联卷积神经网络的方法,该方法提出一种抓取姿态细估计的卷积神经网络模型 Angle-Net,在此基础上,提出一种两阶段级联式抓取位姿检测模型。

% 中文关键词
\keywords{轻量级;卷积神经网络;复值神经网络;级联神经网络;轻量级卷积神经网络;位姿检测模型}
\end{abstract}