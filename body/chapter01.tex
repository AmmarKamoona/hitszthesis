% !TEX root = ../main.tex

% 第1章,中英标题:\chapter{中文标题}[英文标题]
\chapter{绪论}[Introduction]

% 正文内容,注意LaTeX分段有两种方法,直接空一行或者使用<\par>
% 默认首行缩进,不需要在代码编辑区手动敲空格
随着社会发展,人口老龄化,劳动力短缺等问题逐渐凸显,对服务机器人的需求也越来越大,但是服务机器人所工作的非结构化环境也带来了许多技术难题,其中十分主要的一个问题就是非结构环境中机器人的自动抓取,因为抓取是机器人与现实世界交互的主要方式之一。不同于工业机器人在结构化环境中对工件的抓取,服务机器人在非结构化环境下的自动抓取面临着诸多挑战,例如动态化环境、光照变化、物体间存在相互遮挡,以及最主要的,非结构化环境中除了已知的物体,还有大量未知物体。对于非结构化环境中工作的服务机器人,预先获取所有需要进行抓取的物体的模型是不现实的,因此机器人必须能够对未知的物体在线进行快速稳定可靠的抓取规划。

为了解决上述问题,常用的一种做法就是使用机器学习方法来学习从传感器数据提取出的特征表达到抓取位姿的映射关系,相比于建立物体的模型库来保存抓取经验,机器学习方法可以在没见过的物体上进行抓取经验的迁移。在这个领域中,一些传统的方法通常借助于人工设计的特征来表示和存储抓取经验,并训练分类器,但人工设计的特征往往只针对于某一种特定物体或任务有效,且人工设计特征的工作量大难度高,很难在其他场景进行使用。

近年来,以卷积神经网络(Convolutional Neural Network,CNN)为代表的深度学习技术,在计算机视觉和机械设备健康监测等诸多领域取得了重大的突破,在一些领域中达到了超过人类的性能。卷积神经网络可以通过大量数据的训练挖掘出适合于当前任务的特征表达,由于通常卷积神经网络需要堆叠很多层来提高特征表达能力,因此参数较多,需要使用比传统机器学习算法更多的标注数据来进行训练,抑制过拟合提高算法的泛化能力。

在机器人抓取规划领域,使用卷积神经网络学习的特征取代人工设计的特征来对抓取进行表示和分类有很大的优势和应用前景。首先,相比于人工设计的特征,卷积神经网络通过大量数据的学习可以挖掘出泛化能力更强效果更好的特征,可以进一步提高抓取规划算法的性能,且省去了复杂的人工特征设计工作。其次,随着硬件计算力和仿真软件性能的提升,视觉传感器的普及,目前已有许多通过实验或者仿真收集的机器人抓取数据集可供使用。

综上,有了足够的数据以及合理设计的卷积神经网络结构就可以建立更高性能的自动抓取规划算法,进而提高服务机器人在非结构化环境的交互能力,提高其自主化和智能化程度,提高服务机器人的实用性和推进产业落地。因此,基于卷积神经网络的机器人自动抓取规划具有重要的研究价值,可以带来巨大的经济效益。

随着社会发展,人口老龄化,劳动力短缺等问题逐渐凸显,对服务机器人的需求也越来越大,但是服务机器人所工作的非结构化环境也带来了许多技术难题,其中十分主要的一个问题就是非结构环境中机器人的自动抓取,因为抓取是机器人与现实世界交互的主要方式之一。不同于工业机器人在结构化环境中对工件的抓取,服务机器人在非结构化环境下的自动抓取面临着诸多挑战,例如动态化环境、光照变化、物体间存在相互遮挡,以及最主要的,非结构化环境中除了已知的物体,还有大量未知物体。对于非结构化环境中工作的服务机器人,预先获取所有需要进行抓取的物体的模型是不现实的,因此机器人必须能够对未知的物体在线进行快速稳定可靠的抓取规划。

为了解决上述问题,常用的一种做法就是使用机器学习方法来学习从传感器数据提取出的特征表达到抓取位姿的映射关系,相比于建立物体的模型库来保存抓取经验,机器学习方法可以在没见过的物体上进行抓取经验的迁移。在这个领域中,一些传统的方法通常借助于人工设计的特征来表示和存储抓取经验,并训练分类器,但人工设计的特征往往只针对于某一种特定物体或任务有效,且人工设计特征的工作量大难度高,很难在其他场景进行使用。

近年来,以卷积神经网络(Convolutional Neural Network,CNN)为代表的深度学习技术,在计算机视觉和机械设备健康监测等诸多领域取得了重大的突破,在一些领域中达到了超过人类的性能。卷积神经网络可以通过大量数据的训练挖掘出适合于当前任务的特征表达,由于通常卷积神经网络需要堆叠很多层来提高特征表达能力,因此参数较多,需要使用比传统机器学习算法更多的标注数据来进行训练,抑制过拟合提高算法的泛化能力。

在机器人抓取规划领域,使用卷积神经网络学习的特征取代人工设计的特征来对抓取进行表示和分类有很大的优势和应用前景。

首先,相比于人工设计的特征,卷积神经网络通过大量数据的学习可以挖掘出泛化能力更强效果更好的特征,可以进一步提高抓取规划算法的性能,且省去了复杂的人工特征设计工作。

其次,随着硬件计算力和仿真软件性能的提升,视觉传感器的普及,目前已有许多通过实验或者仿真收集的机器人抓取数据集可供使用。

综上,有了足够的数据以及合理设计的卷积神经网络结构就可以建立更高性能的自动抓取规划算法,进而提高服务机器人在非结构化环境的交互能力,提高其自主化和智能化程度,提高服务机器人的实用性和推进产业落地。因此,基于卷积神经网络的机器人自动抓取规划具有重要的研究价值,可以带来巨大的经济效益。
